\chapter{Regular Expressions \label{chapter:regular-expressions}}
\href{http://en.wikipedia.org/wiki/Regular_expression}{\emph{Regular expressions}}
are a very powerful tool when processing strings.
Therefore, most modern programming languages support them.  \setlx\ is no exception.  
As the \setlx\ interpreter is implemented in \textsl{Java}, the syntax of the regular expressions
supported by \setlx\ is the same as the syntax of regular expressions in \textsl{Java}.  For this
reason, instead of discussing every detail of the syntax of regular expressions, we refer the reader to
the documentation of the \textsl{Java} class \texttt{java.util.regex.Pattern}, which can
be found at
\\[0.2cm]
\hspace*{1.3cm}
\href{http://docs.oracle.com/javase/7/docs/api/java/util/regex/Pattern.html}{\texttt{http://docs.oracle.com/javase/7/docs/api/java/util/regex/Pattern.html}}.
\\[0.2cm]
This documentation contains a concise description of the syntax and semantics of regular expressions.
For a more general discussion of the use of regular expression, the book of Friedl
\cite{friedl:2006} is an excellent choice.
Here, we will confine ourselves to show how regular expressions can be used in
\setlx\ programs.  \setlx\ provides two control structures that make use of regular expressions.
The first is the \texttt{match} statement and the second is the \texttt{scan} statement.

\section{Using Regular Expressions in a \texttt{match} Statement}
Instead of the keyword ``\texttt{case}'', a branch in a \texttt{match} statement can begin
with the keyword ``\texttt{regex}''.  Figure \ref{fig:regexp.stlx} shows the definition of
a function named \texttt{classify} that takes a string as its argument
and tries to classify this string as either a word or a number.  


\begin{figure}[!ht]
\centering
\begin{Verbatim}[ frame         = lines, 
                  framesep      = 0.3cm, 
                  firstnumber   = 1,
                  labelposition = bottomline,
                  numbers       = left,
                  numbersep     = -0.2cm,
                  xleftmargin   = 0.8cm,
                  xrightmargin  = 0.8cm,
                ]
    classify := procedure(s) {
        match (s) {
            regex '0|[1-9][0-9]*': print("found an integer");
            regex '[a-zA-Z]+'    : print("found a word");
            regex '\s+'          : // skip white space
            default              : print("unkown: $s$");
        }
    };
\end{Verbatim}
\vspace*{-0.3cm}
\caption{A simple function to recognize numbers and words.}
\label{fig:regexp.stlx}
\end{figure}

Note that we have specified the regular expressions
using literal strings, i.e.~strings enclosed in single quote characters. This is necessary,
since the regular expression in line 5 \\[0.2cm]
\hspace*{1.3cm}
\verb|'\s+'|
\\[0.2cm]
contains a backslash character.  If we had used double quotes, it would have been
necessary to escape the backslash character with another backslash and we would have had
to write
\\[0.2cm]
\hspace*{1.3cm}
\verb|"\\s+"|
\\[0.2cm]
instead.  Invoking the function \texttt{classify} as 
\\[0.2cm]
\hspace*{1.3cm}
\texttt{classify("123");}
\\[0.2cm]
 prints the message ``\texttt{found an integer}'', while invoking the function as
\\[0.2cm]
\hspace*{1.3cm}
\texttt{classify("Hugo");}
\\[0.2cm]
prints the message `\texttt{found a word}''.  Finally, calling
\\[0.2cm]
\hspace*{1.3cm}
\texttt{classify("0123");}
\\[0.2cm]
prints the answer ``\texttt{unkown: 0123}''.  The reason is that the string
``\texttt{0123}'' does not comply with the regular expression \texttt{'0|[1-9][0-9]*'} because this
regular expression does not allow a number to start with the character ``\texttt{0}'' followed by
more digits.  


\subsection{Extracting Substrings}
Often,  strings are structured and the task is to extract substrings corresponding to
certain parts of a string.  This can be done using regular expressions.  For 
example, consider a phone number in the format
\\[0.2cm]
\hspace*{1.3cm}
\texttt{+49-711-6673-4504}.
\\[0.2cm]
Here, the substring ``\texttt{49}'' is the country code, the substring ``\texttt{711}'' is
the area code, the substring ``\texttt{6673}'' is the company code, and finally
``\texttt{4504}'' is the extension.  The regular expression 
\\[0.2cm]
\hspace*{1.3cm}
\verb|\+([0-9]+)-([0-9]+)-([0-9]+)-([0-9]+)|
\\[0.2cm]
specifies this format and the different blocks of parentheses correspond to the different
codes. If  a phone number is given and the task is to
extract, say, the country code and the area code, then this can be achieved with the 
\setlx\ function shown in Figure \ref{fig:extract-phone-code.stlx}.

\begin{figure}[!ht]
\centering
\begin{Verbatim}[ frame         = lines, 
                  framesep      = 0.3cm, 
                  firstnumber   = 1,
                  labelposition = bottomline,
                  numbers       = left,
                  numbersep     = -0.2cm,
                  xleftmargin   = 0.3cm,
                  xrightmargin  = 0.3cm,
                ]
    extractCountryArea := procedure(phone) {
        match (phone) {
            regex '\+([0-9]+)-([0-9]+)-([0-9]+)-([0-9]+)' as [_, c, a, _, _]:
                return [c, a];
            default: abort("The string $phone$ is not a phone number!");
        }
    };
\end{Verbatim}
\vspace*{-0.3cm}
\caption{A function to extract the country and area code of a phone number.}
\label{fig:extract-phone-code.stlx}
\end{figure}

\noindent
Here, the regular expression to recognize phone numbers has several parts that are
enclosed in parentheses.  These parts are collected in a list of the form
\\[0.2cm]
\hspace*{1.3cm}
$[s, p_1, \cdots, p_n]$.
\\[0.2cm]
The first element $s$ of this list is the string that matched the regular expression.  The
remaining elements $p_i$ correspond to the different parts of the regular expression: $p_1$
corresponds to the first group of parentheses, $p_2$ corresponds to the second group, and
in general $p_i$ corresponds to the $i$-th group.  In line 3 this list is then matched
against the pattern
\\[0.2cm]
\hspace*{1.3cm}
\texttt{[\_, c, a, \_, \_]}.
\\[0.2cm]
Therefore, if the match is successful, the variable \texttt{c} will contain the country
code and the variable \texttt{a} is set to the area code.  The groups of
regular expressions that are not needed are matched against the anonymous variable
``\texttt{\_}'' and are therefore effectively ignored.

If a regular expression contains  nested groups of parentheses, then the order of
the groups is determined by the left parenthesis of the group.  For example, the regular
expression 
\\[0.2cm]
\hspace*{1.3cm}
\texttt{'((a+)(b*)c?)d'}
\\[0.2cm]
contains three groups:
\begin{enumerate}
\item the first group is \texttt{'((a+)(b*)c?)'},
\item the second group is \texttt{'(a+)'}, and
\item the third group is \texttt{'(b*)'}.
\end{enumerate}

\subsection{Testing Regular Expressions}
In real life applications, regular expressions can get quite involved and difficult to
comprehend.  The function \texttt{testRegexp} shown in Figure \ref{fig:test-regexp.stlx}
can be used to test a given regular expression:  The function takes a regular expression
\texttt{re} as its first argument, while the second argument is a string \texttt{s}.  The
function tests whether the string \texttt{s} matches the regular expression \texttt{re}.
If this is the case, the function \texttt{testRegexp} returns a list that contains all the
substrings corresponding to the different parenthesized groups of the regular expression
\texttt{re}.   For example,  invoking this function as
\\[0.2cm]
\hspace*{1.3cm}
\texttt{testRegexp('(a*)(a+)b', "aaab");}
\\[0.2cm]
yields the result
\\[0.2cm]
\hspace*{1.3cm}
\texttt{["aaab", "aa", "a"]}.
\\[0.2cm]
Here, the first element of the list is the string that was matched by the regular
expression, the second element \texttt{"aa"} corresponds to the regular subexpression
\texttt{'(a*)'}, and the last element \texttt{"a"} corresponds to the regular
subexpression \texttt{'(a+)'}.

\begin{figure}[!ht]
\centering
\begin{Verbatim}[ frame         = lines, 
                  framesep      = 0.3cm, 
                  firstnumber   = 1,
                  labelposition = bottomline,
                  numbers       = left,
                  numbersep     = -0.2cm,
                  xleftmargin   = 0.8cm,
                  xrightmargin  = 0.8cm,
                ]
    testRegexp := procedure(re, s) {
        match (s) {
            regex re as l: return l;
            default : print("No match!");
        }
    };
\end{Verbatim}
\vspace*{-0.3cm}
\caption{Testing regular expressions.}
\label{fig:test-regexp.stlx}
\end{figure}

\subsection{Extracting Comments from a File}
In this section we present an example of the  \texttt{match} statement in action.  The
function \texttt{printComments} shown in Figure \ref{fig:find-comments.stlx} attempts to
extract all those \texttt{C}-style comments from a file that are contained in a single
line.  The regular expression \verb|'\s*(//.*)'| in line 5 matches comments starting
with the string ``\texttt{//}'' preceded by optional white space, while the regular expression
\\[0.2cm]
\hspace*{1.3cm}
\verb"'\s*(/\*([^*]|\*+[^*/])*\*+/)\s*'"
\\[0.2cm]
in line 6 matches comments that start with the string ``\texttt{/*}'', preceded by optional
white space and end with the string ``\texttt{*/}'' followed by optional trailing white space.  This
regular expression is quite difficult to read for three reasons:
\begin{enumerate}
\item We have to precede the symbol ``\texttt{*}'' with a backslash in order to prevent it
      from being interpreted as a regexp quantifier.
\item We have to ensure that the string between  ``\texttt{/*}'' and ``\texttt{*/}'' does
      not contain the substring ``\texttt{*/}''.  The regular expression
      \\[0.2cm]
      \hspace*{1.3cm}
      \verb"([^*]|\*+[^*/])*"
      \\[0.2cm]
      specifies this substring:  This substring can have an arbitrary number of parts
      that satisfy the following specification:
      \begin{enumerate}
      \item A part may consists of any character different from
            the character ``\texttt{*}''.  This is specified by the regular expression
            \verb"'[^*]'".
      \item A part may be a sequence of ``\texttt{*}'' characters
            that is neither followed by the character ``\texttt{/}'' nor the character
            ``\texttt{*}''.   These parts are matched by the regular expression
            \verb"\*+[^*/]".
      \end{enumerate}
      Concatenating any number of these parts will never produce a string containing the
      substring ``\texttt{*/}''.
\item Lastly, we have to ensure that we can match a string of the form
      ``\texttt{***}$\cdots$\texttt{***/}'' that ends the comment.  The problem here is that
      we need not only be able to recognize the string ``\texttt{*/}'' but also have to deal with
      the case that this string is preceded by an arbitrary number of ``\texttt{*}''-characters,
      since the regular expression \verb"([^*]|\*+[^*/])*" does not accept any trailing
      ``\texttt{*}''-characters. 
\end{enumerate}

\begin{figure}[!ht]
\centering
\begin{Verbatim}[ frame         = lines, 
                  framesep      = 0.3cm, 
                  firstnumber   = 1,
                  labelposition = bottomline,
                  numbers       = left,
                  numbersep     = -0.2cm,
                  xleftmargin   = 0.3cm,
                  xrightmargin  = 0.3cm,
                ]
    printComments := procedure(file) {
        lines := readFile(file);
        for (l in lines) {
            match (l) {
                regex '\s*(//.*)'                       as c: print(c[2]);
                regex '\s*(/\*([^*]|\*+[^*/])*\*+/)\s*' as c: print(c[2]);
            }
        }
    };
    
    for (file in params) {
        printComments(file);
    }
\end{Verbatim}
\vspace*{-0.3cm}
\caption{Extracting comments from a given file.}
\label{fig:find-comments.stlx}
\end{figure}

If the code shown in Figure \ref{fig:find-comments.stlx} is stored in the file
\texttt{find-comments.stlx}, then we can invoke this program as
\\[0.2cm]
\hspace*{1.3cm}
\texttt{setlx find-comments.stlx --params file.stlx}
\\[0.2cm]
The option ``\texttt{--params}'' creates the global variable \texttt{params} that contains
a list containing all the remaining arguments given to the program.   In this case, there is just
one argument, which is the string ``\texttt{file.stlx}''.   Therefore, \texttt{params} is a list
of length one containing the file name as a string.   Hence the \texttt{for} loop in line 11 will call the
function \texttt{printComments} with the given string.  If we invoke the
program using the command
\\[0.2cm]
\hspace*{1.3cm}
\texttt{setlx find-comments.stlx --params *.stlx}
\\[0.2cm]
instead, then the function \texttt{printComments} will be called for all files ending in
``\texttt{.stlx}''. 

\subsection{Conditions in \texttt{match} Statements}
A clause of a \texttt{match} statement can contain an optional Boolean condition that is
separated from the regular expression by the condition operator ``\texttt{|}''.   Figure
\ref{fig:find-palindrome.stlx} shows a program to search for 
\href{http://en.wikipedia.org/wiki/Palindrome}{\emph{palindromes}} 
in a given file.
\begin{enumerate}
\item Line 2 reads the file using the function \texttt{readFile}.  This function returns a list of
      strings where each string corresponds to a line of the file.  

      These lines are then concatenated by the function \texttt{join} into a single string.  In this
      string, the different lines are separated by the newline character ``\texttt{\symbol{92}n}''.

      Finally the resulting string is split into a list of words using the function \texttt{split}.
      The second argument of this function is a regular expression that specifies that the string
      should be split at every non-empty sequence of characters that are not letters.
\item The \texttt{for}-loop in line 4 iterates over all these words and the \texttt{match}-statement
      in line 5 tests every word.
\item Line 6 shows how a condition can be attached to a \texttt{regex} clause. The regular
      expression 
      \\[0.2cm]
      \hspace*{1.3cm}
      \verb|'[a-zA-Z]+'|
      \\[0.2cm]
      matches any number of letters, hence it recognizes words.  However, the string \texttt{c[1]}
      corresponding to the match 
      is only added to the set of palindromes if the predicate \texttt{isPalindrome[c[1])} yields
      \texttt{true}.  The function $\texttt{isPalindrome}(s)$ checks whether $s$ is a palindrome that has
      a length of at least 3.
\end{enumerate}

\begin{figure}[!ht]
\centering
\begin{Verbatim}[ frame         = lines, 
                  framesep      = 0.3cm, 
                  firstnumber   = 1,
                  labelposition = bottomline,
                  numbers       = left,
                  numbersep     = -0.2cm,
                  xleftmargin   = 0.8cm,
                  xrightmargin  = 0.8cm,
                ]
    findPalindrome := procedure(file) {
        all := split(join(readFile(file), "\n"), '[^a-zA-Z]+');
        palindromes := {};
        for (s in all) {
            match (s) {
                regex '[a-zA-Z]+' as c | isPalindrome(c[1]): 
                      palindromes += { c[1] };
                regex '.|\n': 
                      // skip everything else
            }
        }
        return palindromes;        
    };
    
    isPalindrome := procedure(s) {
        if (#s < 3) {
            return false;
        }
        return reverse(s) == s;
    };
\end{Verbatim}
\vspace*{-0.3cm}
\caption{A program to find palindromes in a file.}
\label{fig:find-palindrome.stlx}
\end{figure}


\section{The \texttt{scan} Statement}
The program to extract comments that was presented in the previous subsection is quite
unsatisfactory as it will only recognize those strings that span a single line.
Figure \ref{fig:find-comments-scan.stlx} on page \pageref{fig:find-comments-scan.stlx} shows a
function that instead extracts all 
comments from a given file.  The function \texttt{printComments} takes a string as
argument.  This string is interpreted as the name of a file.  In line 2, the function 
\texttt{readFile} reads this file.  This function produces a list of strings.  Each string
corresponds to a single line of the file without the trailing line break.  The function
\texttt{join} joins all these lines into a single string.  As the second argument of
\texttt{join} is the string ``\texttt{\symbol{92}n}'', a newline is put in between the
lines that are joined.  The end result is that the variable \texttt{s} contains the
content of the file as one string.  This string is then scanned using the \texttt{scan}
statement in line 3. The general form of a scan statement is as follows:
\\[0.2cm]
\hspace*{1.3cm} \texttt{scan ($s$) \{}  \\
\hspace*{1.8cm} \texttt{regex} $r_1$ \texttt{as} $l$ : $b_1$ \\
\hspace*{1.8cm} $\vdots$                                                  \\
\hspace*{1.8cm} \texttt{regex} $r_n$ \texttt{as} $l$ : $b_n$ \\
\hspace*{1.3cm} \texttt{\}}             
\\[0.2cm]
Here, $s$ is a string to be analyzed, $r_1$, $\cdots$, $r_n$ denote regular
expressions, while $b_1$, $\cdots$, $b_n$ are lists of statements.  The \texttt{scan}
statement works as follows:

\begin{figure}[t]
\centering
\begin{Verbatim}[ frame         = lines, 
                  framesep      = 0.3cm, 
                  firstnumber   = 1,
                  labelposition = bottomline,
                  numbers       = left,
                  numbersep     = -0.2cm,
                  xleftmargin   = 0.8cm,
                  xrightmargin  = 0.8cm,
                ]
    printComments := procedure(file) {
        s := join(readFile(file), "\n");
        scan (s) {
            regex '//[^\n]*'                as c: print(c[1]);
            regex '/\*([^*]|\*+[^*/])*\*+/' as c: print(c[1]);
            regex '.|\n'                        : // skip every thing else
        }
    };
\end{Verbatim}
\vspace*{-0.3cm}
\caption{Extracting comments using the match statement.}
\label{fig:find-comments-scan.stlx}
\end{figure}


\begin{enumerate}
\item All the regular expressions $r_1$, $\cdots$, $r_n$ are tried in parallel to match a
      prefix of the string $s$.
      \begin{enumerate}
      \item If none of these regular expression matches, the scan statement is aborted
            with an error message.
      \item If exactly one regular expression $r_i$ matches, then the corresponding
            statements $b_i$ is executed and the prefix matched by $r_i$ is removed from
            the beginning of $s$.
      \item If more than one regular expression matches, then there is a conflict which
            is resolved in two steps:
            \begin{enumerate}
            \item If the prefix matched by some regular expression $r_j$ is longer
                  than any other prefix matched by another regular expression $r_i$,
                  then the regular expression $r_j$ wins, the list of statements $b_j$ is
                  executed and the prefix matched by $r_j$ is removed from $s$.
            \item If there are two (or more) regular expressions $r_i$ and $r_j$ that both
                  match a prefix of maximal length, then the regular expression with the
                  lowest index wins, i.e.~if $i < j$, then $b_i$ is executed.
            \end{enumerate}
      \end{enumerate}
      Finally, the prefix of $s$ that has been matched is removed from $s$.
\item Step 1 is repeated as long as the string $s$ is not empty.  Therefore, a \texttt{scan}
      statement can be viewed as a \texttt{while} loop combined with a \texttt{match} statement.
\end{enumerate}
The clauses in a \texttt{scan} statement can also have Boolean conditions attached.  This
works the same way as it works for a \texttt{match} statement.

The \texttt{scan} statement provides a functionality that is similar to the functionality
provided  by tools like 
\href{http://en.wikipedia.org/wiki/Lex_(software)}{\texttt{lex}} \cite{lesk:1975} or
\href{http://jflex.de}{\textsl{JFlex}} \cite{klein:2009}. 
In order to support this claim, we present an example program that computes the marks of
an exam.  Assume the results of an exam are collected in a text file like the one shown in
Figure \ref{fig:result.txt} on page \pageref{fig:result.txt}.  The first line of this file shows that this is an exam about
algorithms.  The third line tells us
that there are 6 exercises in the given exam and the remaining lines list the points that
have been achieved by individual students.  A hyphen signals that the student did not work
on the corresponding exercise.  In order to calculate marks, we first have to sum up all the points.
From this sum, the mark can then be calculated.

\begin{figure}[!ht]
\centering
\begin{Verbatim}[ frame         = lines, 
                  framesep      = 0.3cm, 
                  firstnumber   = 1,
                  labelposition = bottomline,
                  numbers       = left,
                  numbersep     = -0.2cm,
                  xleftmargin   = 0.8cm,
                  xrightmargin  = 0.8cm,
                ]
    Exam:  Algorithms
    
    Exercises:          1. 2. 3. 4. 5. 6.
    Magnus Peacemaker:   8  9  8  -  7  6
    Dewy Dullamore:      4  4  2  0  -  -
    Alice Airy:          9 12 12  9  9  6
    Jacob James:         9 12 12  -  9  6
\end{Verbatim}
\vspace*{-0.3cm}
\caption{Typical results from an exam.}
\label{fig:result.txt}
\end{figure}


\begin{figure}[!ht]
\centering
\begin{Verbatim}[ frame         = lines, 
                  framesep      = 0.3cm, 
                  firstnumber   = 1,
                  labelposition = bottomline,
                  numbers       = left,
                  numbersep     = -0.2cm,
                  xleftmargin   = 0.8cm,
                  xrightmargin  = 0.8cm,
                ]
    evalExam := procedure(file, maxPoints) {
        all   := join(readFile(file), "\n");
        state := "normal";
        scan (all) using map {
            regex '[a-zA-Z]+:.*\n': // skip header lines
            regex '[A-Za-z]+\s[A-Za-z\-]+:' as [ name ]:
                  nPrint(name);
                  state     := "printBlanks";
                  sumPoints := 0;
            regex '[ \t]+' as [ whiteSpace ] | state == "printBlanks":
                  nPrint(whiteSpace);  
                  state := "normal";
            regex '[ \t]+' | state == "normal": 
                  // skip white space between results
            regex '0|[1-9][0-9]*' as [ number ]:
                  sumPoints += int(number);     // add results
            regex '-': // skip exercises that have not been done  
            regex '\n' | sumPoints != om:       // calculate mark
                  print(mark(sumPoints, maxPoints));
                  sumPoints := om;
            regex '[ \t]*\n' | sumPoints == om: // skip empty lines
            regex '.|\n' as [ c ]:              // error handling
                  print("unrecognized character: $c$");
                  print("line:   ", map["line"]);
                  print("column: ", map["column"]);
        }
    };
    mark := procedure(p, m) {
        return 7.0 - 6.0 * p / m;
    };    
\end{Verbatim}
\vspace*{-0.3cm}
\caption{A program to compute marks for an exam.}
\label{fig:exam.stlx}
\end{figure}

\noindent
Figure \ref{fig:exam.stlx} shows a program that performs this calculation.  We discuss this
program line by line.
\begin{enumerate}
\item The function \texttt{evalExam} takes two arguments:  The first is the name of a file
      containing the results of the exam and the second argument is the number of points 
      needed to get the top mark.  This number is a parameter that is needed to calculate
      the mark of a student.
\item Line 2 creates a string containing the content of the given file.
\item We will use two states in our scanner:
      \begin{enumerate}
      \item The first state is identified by the string \texttt{"normal"}.
            In line 3, the variable \texttt{state} is initialized to this value.
      \item The second state is identified by the string \texttt{"printBlanks"}. 
            We enter this state when we have read the name of a student.
            This state is needed to read the white space between the name of a student 
            and the first number following the student.

            In state \texttt{"normal"}, all white space is discarded, but in state
            \texttt{"printBlanks"} white space is printed.  This is necessary to format
            the output neatly.
      \end{enumerate}
      Line 3 therefore initializes the variable \texttt{state} to \texttt{"normal"}.
\item Line 4 shows the \texttt{scan} statement.  Here, we have added the string ``\texttt{using map}'' 
      after ``\texttt{scan(all)}''.  This enables us to retrieve line and column information via the
      variable \texttt{map}.  How this is done is shown in line 24 and 25.
\item The regular expression in line 5 is needed to skip the header of the file.  The
      two header lines can be recognized by the fact that there is a single word followed by a colon
      ``\texttt{:}''.   In contrast, the names of students always consist of two words separated by
      a single white space character.
\item The regular expression in line 5 matches the name of a student.  Notice that we have used 
      ``\texttt{\symbol{92}s}'' to specify the blank between the first name and the family name.
      Furthermore, the family name is allowed to contain the hyphen character ``\texttt{-}''.
      The student's name is printed and the number of points for this student is set to $0$.
      Furthermore, after the name of a student has been seen, the state is changed to the state
      \texttt{"printBlanks"}.
\item The regular expression in line 10 matches the white space following the name of a
      student.  Notice that this regular expression has a condition attached to it: 
      The condition is the formula
      \\[0.2cm]
      \hspace*{1.3cm}
      \texttt{state == \symbol{34}printBlanks\symbol{34}}.
      \\[0.2cm]
      The white space matched by the regular expression is then printed and the state is switched back to
      \texttt{"normal"}.

      In effect, this rule will guarantee that the output is formatted in the same way as
      the input, as the white space following a student's name is just copied verbatim to the output.
\item Line 13 skips over white space that is encountered in state \texttt{"normal"}.
      This is the white space separating different numbers in the list of numbers following a
      students name.
\item The clause in line 15 recognizes strings that can be interpreted as numbers.  These
      strings are converted into numbers with the help of the conversion function \texttt{int} and
      then this number is added to the number of points achieved by this student.
\item Line 17 skips over hyphens as these correspond to those exercises that have not been attempted
      by the particular student.
\item Line 18 checks whether the end of a line listing the points of a particular student has been reached.
      This is the case if we encounter a newline and the variable \texttt{sumPoints}
      is not undefined.  In this case, the mark for this student is computed and printed.
      Furthermore, the variable \texttt{sumPoints} is set back to the undefined status.
\item Any empty lines are skipped in line 21.
\item Finally, if we encounter any remaining character, then there is a syntax error
      in our input file.  In this case, line 22 recognizes this character and
      produces an error message.  This error message
      specifies the line and column of the character.  This is done with the help of the
      variable \texttt{map} that has been declared in line 4 via the \texttt{using}
      directive.  \texttt{map} is effectively a dictionary that supports the keys \texttt{char},
      \texttt{line}, and \texttt{column}.  
      \begin{enumerate}
      \item \texttt{char} counts the characters encountered, hence this variable is incremented for 
            every character that is read.  This variable has the value $1$ for the first character.
      \item \texttt{line} counts the line numbers.  The line number is incremented once a newline
            character is read. This variable has the value $1$ for the first line.
      \item \texttt{column} counts the characters after the last line break, i.e.~this variable is incremented
            for every character that is read and it resets to 1 after a newline character has been read.
      \end{enumerate}
\end{enumerate}

\section{Additional Functions Using Regular Expressions}
There are four predefined functions that use regular expressions.
\begin{enumerate}
\item The function 
      \\[0.2cm]
      \hspace*{1.3cm}
      $\mathtt{matches}(s, r)$ 
      \\[0.2cm]
      takes a string $s$ and a regular expression $r$ as its arguments.  It returns
      \texttt{true} if the string $s$ is in the language described by the regular
      expression $r$.  For example, the expression
      \\[0.2cm]
      \hspace*{1.3cm}
      \texttt{matches(\symbol{34}42\symbol{34}, '0|[1-9][0-9]*');}
      \\[0.2cm]
      returns \texttt{true} as the string \texttt{\symbol{34}42\symbol{34}} can be
      interpreted as a number and the regular expression \texttt{'0|[1-9][0-9]*'}
      describes natural numbers in the decimal system.

      There is a variant of \texttt{matches} that takes three arguments.  It is called as
      \\[0.2cm]
      \hspace*{1.3cm}
      $\texttt{matches}(s, r, \mathtt{true})$.
      \\[0.2cm]
      In this case, $r$ should be a regular expression containing several \emph{groups},
      i.e.~there should be several subexpressions in $r$ that are enclosed in
      parentheses.  Then, if $r$ matches $s$, the function \texttt{matches} returns a list
      of substrings of $s$.  The first element of this list is  the  string $s$,
      the remaining elements are the substrings corresponding to the different groups of $r$.
      For example, the expression
      \\[0.2cm]
      \hspace*{0.3cm}
      \verb|matches("+49-711-6673-4504", '\+([0-9]+)-([0-9]+)-([0-9]+)-([0-9]+)', true)|
      \\[0.2cm]
      returns the list
      \\[0.2cm]
      \hspace*{1.3cm}
      \texttt{["+49-711-6673-4504", "49", "711", "6673", "4504"]}.
      \\[0.2cm]
      If \texttt{matches} is called with three arguments where the last argument is
      \texttt{true}, an unsuccessful match returns the empty list.
\item The function
      \\[0.2cm]
      \hspace*{1.3cm}
      \texttt{replace($s$, $r$, $t$)}
      \\[0.2cm]
      receives three arguments: the arguments $s$ and $t$ are strings, while $r$ is a
      regular expression.  The function looks for substrings in $s$ that match $r$.  These
      substrings are then replaced by $t$.  For example, the expression
      \\[0.2cm]
      \hspace*{1.3cm}
      \verb|replace("+49-711-6673-4504", '[0-9]{4}', "XXXX");| 
      \\[0.2cm]
      returns the string
      \\[0.2cm]
      \hspace*{1.3cm}
      \verb|"+49-711-XXXX-XXXX"|.
\item There is a variant to the function \texttt{replace($s$, $r$, $t$)} that replaces only the first substring in $s$
      that matches $r$.  This variant is called \texttt{replaceFirst} and is called as
      \\[0.2cm]
      \hspace*{1.3cm}
      \texttt{replaceFirst($s$, $r$, $t$)}.
      \\[0.2cm]
      For example, the expression
      \\[0.2cm]
      \hspace*{1.3cm}
      \verb|replaceFirst("+49-711-6673-4504", '[0-9]{4}', "XXXX");| 
      \\[0.2cm]
      returns the string
      \\[0.2cm]
      \hspace*{1.3cm}
      \verb|"+49-711-XXXX-4504"|.
\item The function
      \\[0.2cm]
      \hspace*{1.3cm}
      \texttt{split($s$, $r$)}
      \\[0.2cm]
      takes a string $s$ and a regular expression $r$ as its inputs.  It returns a list of
      substrings of $s$.  To be more precise, $s$ is split into substrings at those positions that
      match the regular expression $r$.  The parts of $s$ that are matched by $r$ are not returned.
      A common way to invoke the function is given by the following example:
      \\[0.2cm]
      \hspace*{1.3cm}
      \texttt{l := split("1; 2; 3", ';\symbol{92}s*');}
      \\[0.2cm]
      Here, the string \texttt{"1, 2, 3"} is split at every position where there the character
      ``\texttt{;}'' is followed by white space.  As a result, we have
      \\[0.2cm]
      \hspace*{1.3cm}
      \texttt{l = ["1", "2", "3"]}.
      \\[0.2cm]
      If the previous example is changed into
      \\[0.2cm]
      \hspace*{1.3cm}
      \texttt{l := split("1; 2; 3;", ';\symbol{92}s*');}
      \\[0.2cm]
      i.e.~the string is terminated with a ``\texttt{;}'', then the result changes to
      \\[0.2cm]
      \hspace*{1.3cm}
      \texttt{l = ["1", "2", "3", ""]}.
      \\[0.2cm]
      Note that the function 
      \\[0.2cm]
      \hspace*{1.3cm}
      \texttt{join($l$, $s$)}
      \\[0.2cm]
      is an inverse of the function \texttt{split}.  It takes a list of strings $l$ and a string $s$
      and joins the items of $l$ into a single result string by separating them with the string $s$.
      For example,
      \\[0.2cm]
      \hspace*{1.3cm}
      \texttt{join(["a", "b", "c"], ";")}
      \\[0.2cm]
      yields the string ``\texttt{a;b;c}''.
\end{enumerate}




%%% Local Variables: 
%%% mode: latex
%%% TeX-master: "tutorial"
%%% End: 
